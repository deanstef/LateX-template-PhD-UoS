% !TeX root = Background.tex
%% ^^^^^^^^^^^^^^^^^^^^^^^^^^ Change to match title of file
%% ----------------------------------------------------------------
%% Background.tex
%% ---------------------------------------------------------------- 

\documentclass[../main/Progress.tex]{subfiles}
%\documentclass[../main/Thesis.tex]{subfiles}

%TODO uncomment the bib resource if you want autocomplete for references. Remember to comment it out before compiling your main document.
%\addbibresource{../Bibliography/biblio.bib} 

\begin{document}
	\chapter[Background and Literature Review]{Background and \\Literature Review}
	
	\section{Sample section}
	Sample abbreviation \nom{ID}{Identification}
	\subsection{Sample subsection}
	Some text. 
	Change of line.\\
	Sample citation~\cite{Fiat:1987}. 
	A sample image below:
	\begin{figure}[h]
		\centering
		\includegraphics[width=0.5\linewidth]{figure}
		\caption[How the caption will appear in the list of figures]{How the caption will be in text. Here citations can be included \cite{Fiat:1987}.}
		\label{fig:figure}
	\end{figure}
	
This page shows you a subfigure example in \cref{fig:figsubex}.
\begin{figure}[!htb]
  \centering
  \subcaptionbox{The left caption}{
    \includegraphics[width=4.2cm]{figure}
    \label{fig:figsubex:left}
  }
  \subcaptionbox{The right caption}{
    \includegraphics[width=4.2cm]{figure}
    \label{fig:figsubex:right}
  }
  \caption{A doubly colourful picture.}
  \label{fig:figsubex}
\end{figure}


This is how the default reference command \verb|\ref{fig:figure}| outputs cross-references: \ref{fig:figure}.
Using the \verb|cleveref| package, cross-referencing can be smarter. 
Here's the same reference, but using the command \verb|\cref{fig:figure}|: \cref{fig:figure}.
Or, using \verb|\Cref{fig:figure}|: \Cref{fig:figure}.

Template specific instructions can be found at:
\url{https://git.soton.ac.uk/el7g15/uos-latex-template-instructions}.

\Cref{Table:tabex} illustrates the results of some arbitrary example work.
\begin{table}[!htb]
  \centering
  \begin{tabular}{cc}
  \toprule
  \textbf{Training Error} & \textbf{Testing Error}\\
  \midrule
  0 & $\infty$\\
  \bottomrule
  \end{tabular}
  \caption{The Results}
  \label{Table:tabex}
\end{table}
\end{document}